\documentclass{article}
\usepackage{nips_2016}

\setcitestyle{square}

\usepackage{enumerate}
\usepackage{caption}
\usepackage{subcaption}
\usepackage{titlesec}
\usepackage{float}
\usepackage[utf8]{inputenc} % allow utf-8 input
\usepackage[T1]{fontenc}    % use 8-bit T1 fonts
\usepackage{hyperref}       % hyperlinks
\usepackage{url}            % simple URL typesetting
\usepackage{booktabs}       % professional-quality tables
\usepackage{amsfonts}       % blackboard math symbols
\usepackage{nicefrac}       % compact symbols for 1/2, etc.
\usepackage{microtype}      % microtypography
\usepackage{graphicx}
\usepackage{geometry}

\geometry{left = 1cm, right = 1cm, top = 1cm, bottom = 1.75cm}
\titlespacing{\section}{0pt}{9pt}{0pt}

\title{VIX prediction midterm report}
\author{
  Hanyin Cao, Yiyang Wen, Leqi Zhao \\
  Cornell University \\
  \texttt{\{hc936, yw892, lz469\}@cornell.edu} \\
}
\date{October 27, 2017}


\begin{document}

\maketitle

\section*{Abstract}
Uncertainty plays a crucial role in financial analysis, and it is recognized as a complex concept based on its correlations with macroeconomic environment, market expectation and investor's sentiment. Among various methods of describing uncertainty, volatility is the most practical one which gives us a simple way to quantify the uncertainty. Here, we introduce a ticker symbol for the Chicago Board Options Exchange(CBOE) Volatility Index—VIX, often referred to as the “fear index” as it is a measure of the stock market’s expectation of volatility. It is constructed using the implied volatilities of a wide range of S\&P 500 index options.

In this project, we are going to develop an efficient model to predict the direction of VIX, so essentially it is a classification problem with three prediction categories: up, stable and down. We utilized efficient models such as linear regression, ridge regression, lasso and random forests to achieve our goal. 

\section*{Data Overview}
The data we are using can be grouped into three categories:\par
\begin{enumerate}[a]
    \itemsep0em
    \item The main financial market indices which include S\&P 500, Nasdaq, Russell 2000 and Dow Jones Industrial Average. These indices can reflect the domestic market condition to a large degree as they include a wide variety of securities.
    \item Active options contracts indices including ESA index, SPA index and other 11 indices. Essentially, VIX reflects the implied volatility of the market, so the price of the active options contracts may have fundamental influence on VIX.
    \item Macroeconomic data can reflect general condition of macroeconomy, subsequently influence the general expectation and further influence VIX index. The data we find include daily, weekly, monthly and quarterly indices including consumer price index for all urban consumers, crude oil prices, USD/EUR foreign exchange rate and other 24 indicators. 
    \item Sentiment data indicate the market emotion. Such data can be divided into two parts. First parts are the 23 sentiment indices obtained from Bloomberg which could reflect the market emotion from various aspects such as ratios of different asset class, the performance of certain securities, social surveys of the investors’ sentiment. The second parts are VIX related topics searching volume through Google trends. 
    
\end{enumerate}

For more detailed data description including features full name, meaning, frequency, starting and ending dates, please refer the appendix file in our Github folder. 
 
\section*{Data Gathering}	
The main source of our macroeconomic data is  https://fred.stlouisfed.org/. The website includes a wide range of economic series. On the other hand, we obtained our financial and sentiment indices through the Bloomberg terminal on campus. Besides, we chose 50 most relevant keywords with respect to VIX and utilized the python package pytrend to acquire google trends data through API.
 
\section*{Data Preprocessing}
The data we were processing was quite messy, with different time range, different frequency and different ratio of missing data. Before gathering all features, we decided to preprocess our three categories of data separately. Firstly, we convert the date to standard format. In particular, our macro data had different frequency: daily, weekly, monthly, quarterly. Since we wanted our final data in a daily basis, we filled in our lower frequency data value to a corresponding range of time. 

\section*{Data Description}
First, let's have a look at the distribution of each type of data. We drew the violin plot for VIX index and the hist plot for each feature together with its distribution, which are as follows:

\begin{figure}[h]
\centering
\fbox{\includegraphics[width=18cm,height=10.25cm]{distplot.png}}
\caption{Distribution of Features and VIX index}
\end{figure}

\section*{Correlations}
We found that for market index data and active contracts data, some distribution plots were so similar that we decided to conduct correlation analysis to figure out whether there existed strong correlation relationship between them. 

As the figure 2 shows, we found that for market index data, SPX, SPXT, SPY had strong positive correlation. For active contracts data, all features except STEA had correlation of 1. So besides keeping STEA, we eliminated other redundant features and only remained ESA index as it had the most data points.

\begin{figure}[h]
\begin{center}
\fbox{\includegraphics[width=18cm, height=9cm]{correlation_plot.png}}
\caption{Correlation Analysis}
\end{center}
\end{figure}

\section*{Group the VIX data into three classes}
In the real world, what we really care is the trend of the VIX index so that we can trade on them directly to gain profits. In this project we grouped the VIX data into three classes based on the formula listed below. Furthermore, since our VIX data was range from 1/2/1990 to 10/23/2017, we removed earlier dates data on all features to ensure they had the same start date as VIX.
$$
Target = \left\{
\begin{array}{rcl}
1 && \frac{VIX_t}{VIX_{t-1}} >= 0.2\\
0 && \frac{VIX_t}{VIX_{t-1}} \in (-0.2,0.2)\\ 
-1 && \frac{VIX_t}{VIX_{t-1}} <= -0.2
\end{array}
\right.
$$

\section*{Feature Selection}
\subsection*{Missing Value}
As different features had different time ranges, we were interested in those features' missing value ratios. In this project, we took several steps to handle the missing values:
\begin{enumerate}
\itemsep0em
\item We dropped the features whose missing value ratios were larger than 0.5.
\item We dropped the features having low variance that was smaller than 0.1.
\item We imputed the missing values with the former nearest values because of the property of time series data.
\end{enumerate}

\subsection*{Feature Importance}
Based on the current features, we trained a random forest model and ranked the features according to the attribute importance. We decided to select the features with attribute importance larger than 0.008. 

\begin{figure}[h]
\begin{subfigure}{.5\textwidth}
  \includegraphics[width=0.9\linewidth, height=5.5cm]{Missing_Ratio.png}
  \caption{Missing Ratios}
  \label{fig:sub1}
\end{subfigure}
\begin{subfigure}{.5\textwidth}
  \includegraphics[width=0.9\linewidth,height=5.5cm]{features_importances.png}
  \caption{Features Importance}
  \label{fig:sub2}
\end{subfigure}
\caption{Feature Selection}
\label{fig:image3}
\end{figure}

\section*{Baseline Model}
Our preliminary analysis was using extra trees model to train the data from 1/2/1990 to 12/31/2012 and predict the directions of VIX based on the test data from 1/1/2013 to 10/23/2017. The prediction accuracy of this baseline model was 0.56. 

\section*{Next Step...}

\begin{enumerate}
    \itemsep0em
    \item So far we have processed the macroeconomic data, market indices data and active contracts index and added them into the model. In the next step, we will furtherly absorb the sentiment indices and google trends data into the models, improving the prediction ability.
    \item We have gathered some indices data which can be used to reflect the market sentiment, but we need feature engineering upon them. By researching, we will come up with proper way of feature engineering to enhance the quality of input features.
    \item  We have only used extra trees model to predict the trend of VIX. However, other valuable models can be utilized such as random forests, lasso and ridge regression. We will compare the predicting power of different models and analyze on the reasons behind that. 
    
\end{enumerate}

\section*{Reference}
\begin{enumerate}
    \item http://www.investopedia.com/terms/s/sp500.asp
    \item https://en.wikipedia.org/wiki/Macroeconomics
    \item https://en.wikipedia.org/wiki/S\%26P\_500\_Index
\end{enumerate}

\end{document}
